\chapter{Engineering the Liquid Citizen Platform}
\label{ch:ProjectRequirements}
This chapter will distil the efforts of all the previous chapters leading to the concrete design of a  Citizen Science-focused Liquid Democracy Platform from a software engineering perspective.
Based on the \hyperref[ch:Approach]{discussion of the approach} to fulfill the objective of this project, and due to the \hyperref[sec:DiscussionRW]{fact} that no suitable software exists to base the project on, this section will document the initial software engineering process upon which can be built in order to realize a platform that meets the derived \hyperref[sec:Criteria]{criteria} for a Citizen Science-focused Liquid Democracy Platform.\\
Therefore, we start by explicitly documenting the \hyperref[sec:SoftwareRequirements]{software requirements} which we derive from the previous chapters.
We then continue by laying the foundation for a more technical view of the platform by discussing possibilities of its \hyperref[sec:ArchitecturalDesign]{architectural design} before we go into detail on the available options concerning the \hyperref[sec:ApplicationDesign]{application design} in the exemplary setting of a microservices architecture.
The chapter is then concluded by the specification of two protocols, namely the \hyperref[ssec:PublicInteractionProtocoll]{public interaction protocol} and the \hyperref[ssec:VoteTallyingProtocol]{vote tallying protocol}, that revolve around vote and delegation intents posted in the public space.

\todo{Wire the protocols to an argument made earlier that the platform and the delegation/voting as well as vote tallying process are best seperated for data protection reasons}
\todo{Create user stories pertaining to citizen scientists interaction with the platform}
\todo{Enhance user stories}

% narrative: discuss architectural design -> choose microservices
% how to name the section on the actual microservices implementations?



%we will \hyperref[sec:SoftwareRequirements]{derive software requirements} the prototype needs to fulfill, before describing the \hyperref[sec:Implementation]{process of implementing} said prototype.
%Since we base our prototype on a modular architecture around microservices, we will focus predominantly on the \hyperref[ssec:Architecture]{architecture} of the prototype, as well as a detailed description of the implemented \hyperref[ssec:Microservices]{micro services}.
%Then, we define two protocols, namely the \hyperref[ssec:PublicInteractionProtocoll]{public interaction protocol} and the \hyperref[ssec:VoteTallyingProtocol]{vote tallying protocol}, that revolve around vote and delegation intents in the public space.
%Finally, we present the \hyperref[ssec:DemoApplication]{demo application} that showcases the integration of all the previously introduced parts of the platform.
\section{Software Requirements}
\label{sec:SoftwareRequirements}

Every software engineering process begins with a requirements analysis whose execution will differ considerably depending on which software engineering methodology is being used.
The goal of all requirements analysis processes is to identify the requirements towards the envisioned software product from a variety of different angles, i.e. how is the software used by different user roles, how should it interface with existing software, which of these are functional requirements and which are non-functional.

For the sake of brevity and to increase the agility of potential developers of the platform, we chose to follow the scrum methodologies approach to requirements analysis.
In the scrum method, requirements are formulated as so called user stories.
An user story is a statement of the form "As X I want Y." where X is a user role and Y is the detailed description of an activity the given role should be able to perform with the envisioned final software product.
In the scrum methodology, user stories are then broken down into tasks which are assigned to constributors / developers in scrum sprints.

However, we are aiming for an abstract albeit accurate documentation of the Liquid Citizen platform's requirements, not for a step-by-step manual, which we could not provide anyway without the knowledge of final decisions in the compartments of architecture, languages and frameworks used.
To this end, we will now record all requirements that can be extracted from our previous discussion in chapters~\ref{ch:Theory}~and~\ref{ch:Approach} as user stories.
We have assigned each story a priority score between one and ten. Stories scoring 10 are absolutely necessary for the simplest of prototypes, while those on the other end of the spectrum are not necessary for the core functionality at all and could potentially be delivered as updates even after the platform is already used in production.

\userstory{10}
As a \textit{user} I want to initiate a proposition in order start the political process in a certain context of my interest, thereby becoming the \textit{petition author} of the newly initiated proposition.

\userstory{10}
As a \textit{user} and potential \textit{vote accumulator} I want to vote on a proposition in the voting phase.

\userstory{10}
As a \textit{user} I want to delegate my vote to someone else, becoming a \textit{vote delegator} and making that someone a \textit{vote accumulator}.

\userstory{9}
As a \textit{user} I want to support a proposition group in order to propel it to the discussion phase, thereby becoming \textit{proposition group supporter}.

\userstory{9}
 As a \textit{user} I want support a proposition in a discussion phase in order to propel it to the voting phase, thereby becoming a \textit{proposition supporter}.

\userstory{9}
 As a \textit{user} I want to create a discussion entry on a proposition or change request, thereby becoming a \textit{discussion participant} and a \textit{proposition follower}.

\userstory{9}
 As a \textit{proposition author}, I want to modify the contents of my proposition.

\userstory{9}
 As a \textit{user} I want to edit my context subscriptions in my profile, becoming a \textit{context follower} of each context subscribed to.

\userstory{9}
 As an \textit{admin} I want to delete \textit{users}.

\userstory{8}
 As a \textit{user} I want to change my profile data (change password, mail, political affiliations such as party membership) in order to reflect my current my political profile.
 
 \userstory{8}
 As a \textit{user} I want to create an alternative proposition on an existing proposition, putting it in the original propositions proposition group and inheriting its primary context, thereby becoming \textit{alternative proposition author}.

\userstory{8}
 As a \textit{user} I want to see my profile data in order to get an overview of the entries of my profile

\userstory{8}
 As a \textit{user} I want to see the profile of other users in order to get an idea of the political attitudes and affiliations of other users

\userstory{8}
 As a \textit{user} I want to see a list of all the propositions of the context I’m interested in in order to discover relecvant propositions

\userstory{8}
 As a \textit{moderator} I want to be able to resolve context problems and discussion issues.

\userstory{8}
 As an \textit{admin} I want to withdraw privileges from a \textit{moderator}, making him an ordinary \textit{user}.

\userstory{8}
 As a \textit{user} I want to receive notification when other users delegate their vote.

\userstory{8}
 As a \textit{vote delegator} I want receive a notification when my vote power was used in a propositions voting phase.

\userstory{8}
 As a \textit{vote accumulator} I want to receive a notifiation when vote power delegate to me is withdrawn again.

\userstory{7}
 As \textit{proposition follower}, I want to author and submit a change request towards the contents of the followed proposition.

\userstory{7}
 As a \textit{proposition author}, I want to decide about a change request in order to adapt and improve the proposition.

\userstory{7}
 As a \textit{user} I want to give feedback on a proposition change request expressing my approval or rejection of the edits contained within the change request.

\userstory{7}
 As a \textit{user} I want to up- or downvote other users discussion entries.
\userstory{7}
 As a \textit{user} I want to follow a proposition of interest in order to stay updated, thereby becoming a \textit{proposition follower}.

\userstory{7}
 As a \textit{moderator} I want to receive reports on inappropriately set contexts of propositions.

\userstory{6}
 As a \textit{user} I want to report an incorrectly set context of a proposition.

\userstory{6}
 As a \textit{moderator} I want to block users violating the platform rules.

\userstory{6}
 As a \textit{moderator} I want to block proposals violating the platform rules.

\userstory{6}
 As a \textit{proposition author} I want to get notifications on new change requests pertaining to my followed propositions.

\userstory{6}
 As a \textit{context follower} I want to receive notifications about new propositions in the realms of my subscribed to contexts.

\userstory{6}
 As a \textit{user} I want to set a visibility level for each part of my profile data.

\userstory{5}
 As a \textit{proposition author} I want to change and/or update secondary contexts to follow up the discussion of appropriateness of the secondary contexts.

\userstory{5}
 As a \textit{user} I want to sort the list of propositions by multiple criteria (date created, deadline, author, № of supporters, …) in order to see propositions relevant to my criteria.

\userstory{5}
 As a \textit{user} I want to report inappropriate discussion entries to the moderators of the resp. context.

\userstory{5}
 As a \textit{proposition author}, I want to stay informed about the discussion on my proposition.

\userstory{5}
 As a \textit{moderator} I want to discuss/update the context taxonomy in collaboration with other moderators.

\userstory{5}
 As \textit{proposition follower}, I want to receive notifications about updates to my followed propositions like any changes by the proposition author and added, approved or rejected change requests.
\userstory{5}
 As \textit{proposition follower}, I want to receive notifications about the creation of alternative propositions.

\userstory{4}
 As an \textit{admin} I want to see a dashboard of moderation activities and reports.

\userstory{4}
 As an \textit{admin} I want to override moderator’s decisions.

\userstory{4}
 As an \textit{admin} I want to block or unblock users.

\userstory{4}
 As a \textit{user} I want to see the relevant behavior of another user in order to understand their political (Verhaltensspuren) … (Simon, go for it) \unsure{was haben wir damit gemeint?}

\userstory{4}
 As a \textit{user} I want my data to be shielded from other third parties in order preserve my private identity.

\userstory{4}
 As a \textit{discussion participant}, I want to receive notifications when somebody responds to my discussin entries.

\userstory{3}
 As a \textit{user} I want to sign up (securely) in order to use the services of the platform that require a registered account.

\userstory{3}
 As a \textit{user} I want to manage my notification settings.

\userstory{3}
 As a \textit{context follower} I want to receive a notification when the context taxonomy changes/gets updates with effects on a particular context I follow in order to be able to adjust my context subscriptions.

\userstory{3}
 As a \textit{proposition follower} I want to receive notifications of proposition results.

\userstory{2}
 As an \textit{admin} I want to adjust the platform parameters.

\section{Architectural Design}
\label{sec:ArchitecturalDesign}

While user stories are good tools to document expectations on product functions and to some extent also give hints pertaining to the user interface design, they are completely agnostic of a software systems architecture and the different layers and modules that have to work together within the software product in order to be able to achieve the desired functionality specified by the user story.\\
To break down the user stories into small actionable tasks, it is therefore a hard prerequisite for any team of software engineers and developers, no matter how agile, to think about the design of a software system's architecture.

It is quite clear that the Liquid Citizen platform will have to implement a client-server model, because it needs to manage at least a part of the data in a centralized fashion and has to be available to a large number of users over the internet.
However, there are still a number of options available for the architecture of both client and server which we will explore in the following.

\subsection{Client-side Architecture}
\label{ssec:ClientSideArchitecture}
It is a very common pattern in contemporary software development to build a variety of thick, stateful clients that all interface with the same stateless server(s) over an API thats implemented using industry standard web technology such as REST.
Hence, we should not refer to \textit{the} client, because there might be any suitable number of clients, e.g. single page applications rendered in the browser, server-rendered web applications, native mobile clients, native desktop clients, etc.

While it can be argued that clients contribute more to the vision of a given user story than the server, there are other important techniques suited towards UX design that need to accompany user stories in order to develop a modern client, which we do not see to fall in scope of this work.
We will therefore only concern ourselves with the server side of the platform.

\subsection{Server-side Architecture}
\label{ssec:ServerSideArchitecture}


%% We need layers to draw the block diagram
\pgfdeclarelayer{background}
\pgfdeclarelayer{foreground}
\pgfsetlayers{background,main,foreground}

% Define a few styles and constants
\tikzstyle{insideBlock}=[draw, fill=white!20, text width=5em, 
    text centered, minimum height=2.5em]
\tikzstyle{layer} = [insideBlock, text width=6em, fill=white!20, 
    minimum height=20em, rounded corners]
\def\blockdist{2.3}
\def\edgedist{2.5}

\begin{tikzpicture}
    
    % Note the use of \path instead of \node at ... below. 
    \draw node (c1) [insideBlock] {Client (user)};
%     \path (clientLayerBlock.-150)+(-\blockdist,0) node (c2) [insideBlock] {Client (user)};
%     \path (clientLayerBlock.430)+(-\blockdist,0) node (cn) [insideBlock] {Client / user};
	\path node [below=1em of c1] (c2) [insideBlock] {Client (user)};
	\path node [below=3em of c2] (cn) [insideBlock] {Client (user)};
    \path (c2) -- node (dotdot) {\vdots} (cn);
    \draw node [layer, right=5em of dotdot] (serviceLayer) {Service layer};
    % Unfortunately we cant use the convenient \path (fromnode) -- (tonode) 
    % syntax here. This is because TikZ draws the path from the node centers
    % and clip the path at the node boundaries. We want horizontal lines, but
    % the sensor and naveq blocks aren't aligned horizontally. Instead we use
    % the line intersection syntax |- to calculate the correct coordinate
    \path [draw, ->] (c1) -- node [above] {}
        (serviceLayer.west |- c1) ;
    % We could simply have written (gyros) .. (naveq.140). However, it's
    % best to avoid hard coding coordinates
    \path [draw, ->] (c2) -- node [above] {}
        (serviceLayer.west |- c2);
    \path [draw, ->] (cn) -- node [above] {}
        (serviceLayer.west |- cn);
    \node (clientLayerLabel) [below=1em of cn,align=center] {Client layer \\ (users)};
    
    % Now it's time to draw the colored IMU and INS rectangles.
    % To draw them behind the blocks we use pgf layers. This way we  
    % can use the above block coordinates to place the backgrounds   
    \begin{pgfonlayer}{background}
        % Compute a few helper coordinates
        \path (c1.west |- serviceLayer.north)+(-0.7,0.5) node (a) {};
        \path (c1.north west)+(-0.3,0.3) node (a) {};
        \path (clientLayerLabel.south -| c1.east)+(+0.4,-0.4) node (b) {};
        \path[fill=white!10,rounded corners, draw=black!50, dashed]
            (a) rectangle (b);
    \end{pgfonlayer}
\end{tikzpicture}

%As highlighted in the discussion above, we aimed for a software design that allowed for flexibility, incorporating multiple contexts and a decentralized application context / paradigm. Because of this we chose an approach centered around microservices and (as far as possible) independent modules. Since we want it to be available for many different contexts, we focus on the mechanisms and provide an architecture where the users are independent from the servers offering the services as well as the frontend. For the same reason we also don't offer a frontend solution for the implemented mechanisms, and leave the implementation of fullgrown UIs to designers of applications of the software.


%As mentioned in the introduction of this section, the system architecture of Liquid Citizen is highly decentralized and modular.
%Different components are realized self-sufficiently, with their own APIs, in their own docker containers, and are developed in their respective \href{https://github.com/Liquid-Citizen}{github repositories}.
%Since the different components operate on distinct ports, they can be run on the same machine; However, to be true to the decentralized standard of the system, they can run on different machines just as well.
%Thus their development as well as their production is entirely encapsulated within itself.

%Since the components however form a service ecosystem where the different components need to interact / exchange data with one another, care needs to be taken to integrate the components well with one another, which violates their independence, but is needed for functionality. These aspects will be discussed in the following, in the \hyperref[ssec:Microservices]{description of the microservices}. The respective components, as modules of the software, are sketched \hyperref[ssec:Modules]{afterwards}.


\section{Exemplary Application Design for Microservices}
\label{sec:ApplicationDesign}

 \subsection{Proposition Server}

 \paragraph*{POST: createProposition()} \mbox{} \\
 Creates a new proposition according to the parameters transmitted, as well as the proposition group alternative propositions to the newly created proposition will belong to.

 Transmitted parameters conflicting with the proposition configuration of the proposition service will be ignored.

 \paragraph*{POST: createAlternativeProposition(propositionGroupID)} \mbox{} \\
 Creates a new alternative proposition belonging to the propositionGroupID of the referred proposition.

 Transmitted parameters conflicting with the proposition configuration of the proposition service will be ignored.

 \paragraph*{POST: createPropositionToken(propositionID,plainTextOption)} \mbox{} \\
 Creates a propositionToken for the respective proposition, associated with the respective plainTextOption (as long as the proposition and the option exists). The propositionToken can be used to derive the plainTextOption associated with it, and is introduced so the plainTextOption can't be manipulated by other actors in the vote evaluation process.

 \paragraph*{POST: derivePlainTextOption(propositionToken)} \mbox{} \\
 Returns the plain text option associated with the proposition token.


 \paragraph*{GET: getPropositionInformation(propositionID, foobar)} \mbox{} \\
 Will return relevant information about the proposition in question.

 \paragraph*{GET: getPropositionVoteClosingTime(propositionID)} \mbox{} \\
 Returns the vote closing time of the proposition (when the voting period ended).

 \paragraph*{GET: getPropositionVoteOpeningTime(propositionID)} \mbox{} \\
 Returns the vote opening time of the proposition (when the voting period started).

 \paragraph*{POST: signVote(vote)} \mbox{} \\
 Signs a vote with the current time stamp to prove that the vote was casted on time. 

 \paragraph*{Data base structure proposition service} \mbox{} \\
 The proposition service stores the data for the propositions as well as the proposition tokens used for vote casting. For this it features two tables, one with the proposition data (to be specified) and one with a mapping of the optionTokens and the respective plainTextOptions.




\section{Public Interaction Protocol}
\label{ssec:PublicInteractionProtocol}

\section{Vote Tallying Protocol}
\label{ssec:VoteTallyingProtocol}

%\section{Demo Application}
%\label{ssec:DemoApplication}
