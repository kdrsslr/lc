\chapter{Introduction}
\label{ch:Introduction}

Liquid Democracy (\tracknshrink{LD}) describes a compound of two fundamental types of democracy, that is (1) direct or pure democracy and (2) representative or indirect democracy. 

Liquid democracy empowers citizens to delegate their vote to another citizen. 


\todo[inline]{Quick summary how CS and LD and connected and why this is important (1 paragraph)}

This document is the project report of the LDIG project group, a project in the \href{http://www.dh.uni-leipzig.de/wo/courses/summer-semester-20142015/citizen-science/}{\textit{Citizen Science}} course of the \href{http://www.dh.uni-leipzig.de/wo/}{department of Digital Humanities} at the University Leipzig in the summer semester of 2017.

\section{Motivation}
\label{sec:Motivation}

\todo[inline]{Catchy Aufmacher \& why Citizen Science is important.}

\todo[inline]{Catchy Aufmacher \& why Liquid Democracy is important.}

However, despite a few organizations (ref) using them for participative processes, no platform allowing for processes that truly allow for Liquid Democracy exists (as we will show in \ref{sec:RelatedWork}, the tools we found are basically glorified voting tools). 

Furthermore, existing tools do not explicitly and readily make relevant scientific data transparent, and don't provide even the slightest assistance to citizens interested in understanding the processes taking place through them. 

We believe that it is time for a platform that truly allows for providing the structure for liquid democratic processes that are comprehensible for the centrepiece of a democratic system, the mature, participative Citizen.


\section{Objective}
\label{sec:Objective}

\todo[inline]{Framework wie man Grundsätze einer Demokratie wahrn, reale Abbildung gesellschaftlicher Strukturen. Forschungsfragen stellen aus Daten und aus Community(!)}

As noted in the \hyperref[sec:Motivation]{motivation}, we believe in the importance of Citizen Science enabling liquid democracy within a participative system. Due to the transformation of processes of all kind into the digital world and more and more social context taking place in-silico, we believe this needs to take place through a participatory, decentralized and open online-platform.

The need for participation in societal processes becomes ever more apparent in light of the current developments of our political system. Ever more people feel disenfranchised, with little opportunity to make their voice heard on a political stage. Paradoxically, this development finds its expression in governments that allow for less participation and more centralized structures.

We believe that this not merely a technical problem, in the sense of lacking digital infrastructure; However we think that a societal movement empowering 'the citizen' can profit from a participatory, discursive platform.

As will be shown in \ref{ch:Conclusion}, such a platform does not exist. Our goal is thus to develop technological components, thoroughly grounded in the theory of both \hyperref[sec:Theory_CS]{Citizen Science} and \hyperref[sec:Liquid_Democracy]{Liquid Democracy} that provide a basis for interested citizens to develop such platforms.

These technological components need to enable the citizen to both participate in democratic processes in a way that respects its nature as mature political subject, as well as in enabling them to approach it as an object of research from a scientific perspective, even without being an expert in either the subject matter or research methods. 

It thus needs to provide the functionalities to exhibit the processes of a fully-functional, comprehensive liquid democracy process, as well as the ease of access and learning resources, as well as research infrastructure to perform Citizen Science that deserves its name.

Due to the scope, as well as legal and ethical requirements such fully-fledged Liquid Democracy platform would have to fulfill, we chose to take a more exploratory approach. 

Instead of creating a comprehensive real-world tool, we strive to develop a more playful approach, where certain aspects, mechanism and processes can be explored in a safe environment. For this we chose to make this a game, since we believe that this approach will give insights into how more mature approaches can be designed. Also we believe that a perspective of man as homo ludens will influence the behavior of its users to behave in richer ways and to give rise to more interesting data sets for citizen scientists to analyze.

Thus the objective of this project is to create a game exemplifying aspects and processes in Liquid Democracy with the a strong sensibility for questions relevant in Citizen Science. We want to open the possibility for a large range of researchers to observe the processes in these dynamic policy formation mechanisms, without compromising the privacy and security of the participants.

\section{Structure}
\label{sec:structure}


After \hyperref[sec:Motivation]{motivating} the context of this research and \hyperref[sec:Objective]{formulating the research question}, the \hyperref[ch:Theory]{next section} will give discuss more thoroughly the two major societal contexts this project is situated in, namely \hyperref[sec:Theory_CS]{Citizen Science} and \hyperref[sec:Liquid_Democracy]{Liquid Democracy}. From the discussion of these theoretic aspects, we will derive a \hyperref[sec:Criteria]{range of criteria} a Citizen Science-based Liquid Democracy platform needs to fulfill.

Subsequently, we will explore existing approaches for Liquid Democracy platforms (or voting platforms as is often the case), presenting the current state of the art. We will evaluate in how far these platforms conform to the \hyperref[sec:Criteria]{criteria derived from the theory}. 

As will be shown, no existing approach is suited to fulfill the criteria. Thus, in \hyperref[ch:ProjectRequirements]{the following chapter} we will develop a platform based on the requirements sketched before. This will be done by \hyperref[sec:SoftwareRequirements]{compiling product requirements} the platform needs to fulfill in order to meet the projects requirements. Subsequently, we will explore options in the realms of \hyperref[sec:ArchitecturalDesign]{architectural design}, \hyperref[sec:ApplicationDesign]{application design} and provide protocols for (1) \hyperref[sec:PublicDeclarationOfIntentProtocol]{verifiable voting and delegation in a public space} and (2) \hyperref[sec:VoteTallyingProtocol]{vote tallying in the public space}.

Finally, we will \hyperref[ch:Conclusion]{conclude} by critically discussing the \hyperref[sec:DiscussionImplementation]{prototype implementation} in the context of the project, with a particular focus on the \hyperref[sec:Criteria]{criteria derived from the theory}, before we will conclude with a discussion of \hyperref[sec:FutureWork]{future work}.