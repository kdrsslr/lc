\usepackage{hyperref}	% SOLLTE ALS LETZTES GELADEN WERDEN!

% Printversion:		colorlinks=false und linktoc=all
% Digitalversion:	colorlinks=true	 und linktoc=page	

\hypersetup{
%	bookmarks=true,         % show bookmarks bar?
	unicode=true,          % non-Latin characters in Acrobat’s bookmarks
	pdftoolbar=false,        % show Acrobat’s toolbar?
	pdfmenubar=false,        % show Acrobat’s menu?
	pdffitwindow=false,     % window fit to page when opened
	pdfstartview={FitH},    % fits the width of the page to the window
	pdftitle={Liquid Citzen},    % title
	pdfauthor={Dreßler K., Johannig S., Kalinke S., Kruckenberg C.},     % author
	pdfsubject={Project Report},   % subject of the document
	pdfcreator={Dreßler K., Johannig S., Kalinke S., Kruckenberg C.},   % creator of the document
	pdfproducer={Dreßler K., Johannig S., Kalinke S., Kruckenberg C.}, % producer of the document
	pdfkeywords={Citizen Science, Liquid Democracy}, % list of keywords
	pdfnewwindow=true,      % links in new PDF window
	colorlinks=true,       % false: boxed links; true: colored links
	pdfborder={0 0 0},		% color of the border around links
	linkbordercolor={0 0 0},
	citebordercolor={0 0 0},
	urlbordercolor={0 0 0},
	linktoc=page,				% defines which part of an entry in the table of contents is made into a link (none,section,page,all)
	linkcolor=MidnightBlue,          % color of internal links (change box color with linkbordercolor)
	citecolor=MidnightBlue,        % color of links to bibliography
	filecolor=MidnightBlue,      % color of file links
	urlcolor=MidnightBlue           % color of external links
}

% Nach Hyperref – Glossaries
%\usepackage[nomain,acronym,toc,indexonlyfirst]{glossaries} % Optional: nopostdot https://de.sharelatex.com/learn/Glossaries
%\makeglossaries	% https://en.wikibooks.org/wiki/LaTeX/Glossary
\usepackage{cleveref}% Has to be loaded after hyperref