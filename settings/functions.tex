\usepackage{scrhack}
\usepackage[hyphens]{url}	% allows long URL to break

\usepackage[usenames,dvipsnames,svgnames,table]{xcolor} % https://en.wikibooks.org/wiki/LaTeX/Colors

\usepackage{float}		% enables H for tab/figs: \begin{table}[H]
% if you don't have a blank line after \end{xxx}[H] there will be no indentation:
\makeatletter

\renewcommand\float@endH{\@endfloatbox\vskip\intextsep
  \if@flstyle\setbox\@currbox\float@makebox\columnwidth\fi
  \box\@currbox\vskip\intextsep\relax\@doendpe}

\makeatother


\usepackage{tikz} % advanced grpahics such as figure numbering etc.
\newcommand*\circled[1]{\tikz[baseline=(char.base)]{
		\node[shape=circle,draw,inner sep=1.75pt] (char) {#1};}} % make circled numbers with \circled{1} etc
        
\newcommand\tab[1][1cm]{\hspace*{#1}} % allows to use \tab to create horizontal space


\usepackage{relsize} % wird für \textscale verwendet
\newcommand*{\tracknshrink}[1]{\textscale{0.9}{\textls*[25]{#1}}} % \tracknshrink{UNO}: verkleinert und sperrt Versalabkürzungen


\usepackage{enumitem}	% allows you to change the left margin on e.g. \begin{enumerate}[leftmargin=2cm]

\usepackage{caption}				% fixt den scrollbug bei ref-Links für captions (fig & tables)

\usepackage{graphicx}			% benötigt für Grafiken
\graphicspath{{./img/}}			% global graphics path

\usepackage{booktabs}			% See myriad of alternatives: http://tex.stackexchange.com/questions/12672/which-tabular-packages-do-which-tasks-and-which-packages-conflict
% usage: http://www.namsu.de/Extra/pakete/Booktabs.html
\newcommand*\rot{\rotatebox{90}} % use \rot to rotate text 90° for tables e.g.

\usepackage{csquotes}			% glossaries needs this


\usepackage{xargs}                      % Use more than one optional parameter in a new commands
\usepackage[colorinlistoftodos,prependcaption,bordercolor=yellow!75,backgroundcolor=yellow!40,linecolor=yellow!75,textsize=scriptsize,textwidth=1.95cm]{todonotes}
\newcommandx{\unsure}[2][1=]{\todo[linecolor=Turquoise!30,backgroundcolor=Turquoise!20,bordercolor=Turquoise!30,#1]{#2}}
\newcommandx{\wording}[2][1=]{\todo[linecolor=blue!30,backgroundcolor=blue!20,bordercolor=blue!30,#1]{#2}}
\newcommandx{\info}[2][1=]{\todo[linecolor=Goldenrod!30,backgroundcolor=Goldenrod!20,bordercolor=Goldenrod!30,#1]{#2}}
\newcommandx{\citethis}[2][1=]{\todo[linecolor=BrickRed!30,backgroundcolor=BrickRed!20,bordercolor=BrickRed!30,#1]{#2}}